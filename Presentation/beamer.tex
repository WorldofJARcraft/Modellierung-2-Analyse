	% Darmstadt|Frankfurt|JuanLesPins
	
	\documentclass{beamer}
	\setbeamertemplate{navigation symbols}{}
	
	\usetheme{hpi}
	\usepackage[ngerman]{babel}
	\usepackage[utf8]{inputenc}
	\usepackage{geometry}
	\usepackage[T1]{fontenc}
	\usepackage{graphicx} %Zum Bilder einbinden
	\usepackage{float} %Damit die Figures, also die Bilder, mitten im Text, an geforceter Position erscheinen
	\usepackage{verbatim} %für mehrzeilige Kommentare \begin~\end{comment}
	\usepackage{amstext} % \text{asdf} in Formeln, statt \mbox, weil mbox die Schriftgröße festsetzt	
	\usepackage{amsmath}
	\usepackage{amssymb}
	\usepackage{ucs}
	\usepackage{BeamerColor}
	
	\usepackage{listings}
	\usepackage{color}
	\usepackage{hyperref}
	
	\definecolor{tplcolor}{HTML}{F6AE15}
	\usecolortheme[named=tplcolor]{structure}
	
	\lstset{
		language=Java, 
		inputencoding=utf8,
		tabsize=2,
		basicstyle=\tiny,
		captionpos=b,language=JAVA,breaklines=true,      % the size of the fonts that are used for the line-numbers,
		stepnumber=5,   
		keywordstyle=\color{brown},
		commentstyle=\color{DarkGreen}, 
		stringstyle=\color{blue},
		showstringspaces=false,
		breaklines=true
		literate=%
		{Ö}{{\"O}}1
		{Ä}{{\"A}}1
		{Ü}{{\"U}}1
		{ß}{{\ss}}1
		{ü}{{\"u}}1
		{ä}{{\"a}}1
		{ö}{{\"o}}1
		{~}{{\textasciitilde}}1	
	}

	\beamersetuncovermixins{\opaqueness<1>{25}}{\opaqueness<2->{15}}
	
	\usecaptiontemplate{
	\tiny
	\structure{\insertcaptionname~\insertcaptionnumber:}
	\insertcaption
	}
	
	\usefootnotetemplate{
	\tiny
	\parindent 1em\noindent
	\hbox to 1.8em{\hfil\insertfootnotemark}\insertfootnotetext
	}
	\begin{document}
			
	\setbeamercovered{invisible}
	
	\title[Review Analyseprojekt]{Review Analyseprojekt}
	\author[Eric Ackermann]{Eric Ackermann\\ Gruppe 3}
	
	 \begin{frame}[title=Hauptgebaeude_Nacht.jpg]
	 \maketitle
 	\end{frame}
	 
	\begin{frame}
		\frametitle{Gliederung}
		\tableofcontents
	\end{frame}

\section{Zielbestimmung}		
\begin{frame}
\frametitle{Zielbestimmung}
BONUSPUNKTE!
\end{frame}

\section{Produkteinsatz}		
\begin{frame}
\frametitle{Produkteinsatz}
BONUSPUNKTE!
\end{frame}

\subsection{Beschreibung des Problembereichs}		
\begin{frame}
\frametitle{Beschreibung des Problembereichs}
BONUSPUNKTE!
\end{frame}

\subsection{Glossar}		
\begin{frame}
\frametitle{Glossar}
BONUSPUNKTE!
\end{frame}

\subsection{Modell des Problembereichs}		
\begin{frame}
\frametitle{Modell des Problembereichs}
BONUSPUNKTE!
\end{frame}

\subsection{Geschäftsprozesse}		
\begin{frame}
\frametitle{Geschäftsprozesse}
BONUSPUNKTE!
\end{frame}

\section{Produktfunktion}		
\begin{frame}
\frametitle{Produktfunktion}
BONUSPUNKTE!
\end{frame}

\section{Produktfunktion}		
\begin{frame}
\frametitle{Produktfunktion}
BONUSPUNKTE!
\end{frame}

\subsection{Anwendungsfälle im Überblick}		
\begin{frame}
\frametitle{Anwendungsfälle im Überblick}
BONUSPUNKTE!
\end{frame}

\subsection{Beschreibung der Anwendungsfälle}		
\begin{frame}
\frametitle{Produktfunktion}
BONUSPUNKTE!
\end{frame}

\section{Quellen}
\begin{frame}
		\frametitle{Quellen}
		\begin{itemize}
		  \item kleine graue und weiße Zellen
		\end{itemize}
	\end{frame}
\end{document}
			
