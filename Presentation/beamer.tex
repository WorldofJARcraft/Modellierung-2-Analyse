	% Darmstadt|Frankfurt|JuanLesPins
	
	\documentclass{beamer}
	\setbeamertemplate{navigation symbols}{}
	
	\usetheme{hpi}
	\usepackage[ngerman]{babel}
	\usepackage[utf8]{inputenc}
	\usepackage{geometry}
	\usepackage[T1]{fontenc}
	\usepackage{graphicx} %Zum Bilder einbinden
	\usepackage{float} %Damit die Figures, also die Bilder, mitten im Text, an geforceter Position erscheinen
	\usepackage{verbatim} %für mehrzeilige Kommentare \begin~\end{comment}
	\usepackage{amstext} % \text{asdf} in Formeln, statt \mbox, weil mbox die Schriftgröße festsetzt	
	\usepackage{amsmath}
	\usepackage{amssymb}
	\usepackage{ucs}
	\usepackage{BeamerColor}
	
	\usepackage{listings}
	\usepackage{color}
	\usepackage{hyperref}
	\usepackage{acronym}
	\definecolor{tplcolor}{HTML}{F6AE15}
	\usecolortheme[named=tplcolor]{structure}
	
	\lstset{
		language=Java, 
		inputencoding=utf8,
		tabsize=2,
		basicstyle=\tiny,
		captionpos=b,language=JAVA,breaklines=true,      % the size of the fonts that are used for the line-numbers,
		stepnumber=5,   
		keywordstyle=\color{brown},
		commentstyle=\color{DarkGreen}, 
		stringstyle=\color{blue},
		showstringspaces=false,
		breaklines=true
		literate=%
		{Ö}{{\"O}}1
		{Ä}{{\"A}}1
		{Ü}{{\"U}}1
		{ß}{{\ss}}1
		{ü}{{\"u}}1
		{ä}{{\"a}}1
		{ö}{{\"o}}1
		{~}{{\textasciitilde}}1	
	}

	\beamersetuncovermixins{\opaqueness<1>{25}}{\opaqueness<2->{15}}
	
	\usecaptiontemplate{
	\tiny
	\structure{\insertcaptionname~\insertcaptionnumber:}
	\insertcaption
	}
	
	\usefootnotetemplate{
	\tiny
	\parindent 1em\noindent
	\hbox to 1.8em{\hfil\insertfootnotemark}\insertfootnotetext
	}
	\begin{document}
			
	\setbeamercovered{invisible}
	
	\title[Review Analyseprojekt]{Review Analyseprojekt}
	\author[Eric Ackermann]{Eric Ackermann\\ Gruppe 3}
	
	 \begin{frame}[title=Hauptgebaeude_Nacht.jpg]
	 \maketitle
 	\end{frame}
	 
	\begin{frame}
		\frametitle{Gliederung}
		\tableofcontents
		Folien mit einem * im Titel sollen beim Vortrag übersprungen werden.
	\end{frame}

\section{Produkteinsatz}		
\begin{frame}
\frametitle{Produkteinsatz*}
Im Folgenden werden die notwendigen Fachbegriffe und Zusammenhänge näher erläutert. Des Weiteren werden systemrelevante Ablaufe im Einsatzbereich dargestellt und die erläuterten Fachbegriffe in Beziehung zu diesen Abläufen gesetzt.\\
Dafür wird bewusst ein IST-Zustand der aktuellen Situation aufgeführt, um daraus den Einsatz des neuen Produktes (Soll-Zustand) herzuleiten.
\end{frame}

\subsection{Beschreibung des Problembereichs}		
\begin{frame}
\frametitle{Beschreibung des Problembereichs*}
Im Rahmen der Aufgabenstellung wird hier auf eine Problembereichs-Beschreibung verzichtet. Allerdings sind folgende Annahmen über den Problembereich getroffen worden:
\begin{itemize}
\item Stammdaten können geändert werden, allerdings sind manche dieser Daten (z.B. Kontodaten) bei einer Änderung zu verifizieren. Andere wiederum können ohne Verifizierung direkt geändert werden.
\item Einsatzleiterausbildungen sind Qualifikationen, auch wenn sie im gegebenen Beispiel als eigene Spalte aufgeführt werden.
\item Fähigkeiten eines Helfers, die nicht urkundlich nachgewiesen werden müssen, werden nicht in der Personalverwaltung aufgeführt
\end{itemize}
\end{frame}

\subsection{Glossar}		
\begin{frame}
\frametitle{Glossar*}
\begin{acronym}[Qualifikation]  
	\acro{Stammdaten}{Personenbezogene und persönliche Daten, die direkt beim Eintritt erfasst werden. Diese umfassen Namen, Vornamen, Geburtsdaten, Adresse, Telefon, Mailadresse und evtl. Bankdaten des jeweiligen Helfers.} 
	\acro{Helfer}{Synonym zu Ehrenamtlicher, alle erfassten Mitglieder, die nicht hauptamtlich tätig sind. Bezieht sich nicht auf die Fähigkeit, an Sanitätseinsätzen o.Ä. teilnehmen zu können.}  
	
\end{acronym}
\end{frame}

\begin{frame}
\frametitle{Glossar*}
\begin{acronym}[Führungskraft]  
	\acro{Qualifikation}{Urkundlich nachweisbare Befähigung eines Helfers. Dies Befähigung kann einerseits den Helfer zu mehr Aufgabenbereichen befähigen (wie das Führen eines Fahrzeugs) andererseits auch verpflichtend sein, um weiterhin Helfer zu sein (wie die Arbeitsmedizinische Untersuchung). Qualifikationen können unbefirstet gültig sein, sie können aber auch nach einer gewissen Dauer ablaufen, und sie müssen erneut nachgewiesen werden.
	}
	\acro{Führungskraft}{Helfer mit einer Funktion\\ Helfer mit einer Funktion
		Funktion (eines Helfers): Eine von fünf Rollen (Personalbeauftragter, Neuhelferbeauftragter, Ausbildungsbeauftragter, Einsatzkoordinator, Fahrzeugbeauftragter), von den immer nur genau eine immer nur genau einem Helfer übernommen werden kann.}  
\end{acronym}
\end{frame}
\subsection{Modell des Problembereichs}		
\begin{frame}
\frametitle{Modell des Problembereichs}
BONUSPUNKTE!
\end{frame}

\subsection{Geschäftsprozesse}		
\begin{frame}
\frametitle{Geschäftsprozesse}
BONUSPUNKTE!
\end{frame}

\section{Produktfunktion}		
\begin{frame}
\frametitle{Produktfunktion}
BONUSPUNKTE!
\end{frame}

\section{Produktfunktion}		
\begin{frame}
\frametitle{Produktfunktion}
BONUSPUNKTE!
\end{frame}

\subsection{Anwendungsfälle im Überblick}		
\begin{frame}
\frametitle{Anwendungsfälle im Überblick}
BONUSPUNKTE!
\end{frame}

\subsection{Beschreibung der Anwendungsfälle}		
\begin{frame}
\frametitle{Produktfunktion}
BONUSPUNKTE!
\end{frame}

\section{Quellen}
\begin{frame}
		\frametitle{Quellen}
		\begin{itemize}
		  \item kleine graue und weiße Zellen
		\end{itemize}
	\end{frame}
\end{document}
			
