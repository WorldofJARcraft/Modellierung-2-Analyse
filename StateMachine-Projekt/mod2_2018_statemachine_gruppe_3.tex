	% Darmstadt|Frankfurt|JuanLesPins
	
	\documentclass{beamer}
	\setbeamertemplate{navigation symbols}{}
	
	\usetheme{hpi}
	\usepackage[ngerman]{babel}
	\usepackage[utf8]{inputenc}
	\usepackage{geometry}
	\usepackage[T1]{fontenc}
	\usepackage{graphicx} %Zum Bilder einbinden
	\usepackage{float} %Damit die Figures, also die Bilder, mitten im Text, an geforceter Position erscheinen
	\usepackage{verbatim} %für mehrzeilige Kommentare \begin~\end{comment}
	\usepackage{amstext} % \text{asdf} in Formeln, statt \mbox, weil mbox die Schriftgröße festsetzt	
	\usepackage{amsmath}
	\usepackage{amssymb}
	\usepackage{ucs}
	\usepackage{BeamerColor}
	
	\usepackage{listings}
	\usepackage{color}
	\usepackage{hyperref}
	\usepackage{acronym}
	\definecolor{tplcolor}{HTML}{000000}
	\usecolortheme[named=tplcolor]{structure}
	
	\lstset{
		language=Java, 
		inputencoding=utf8,
		tabsize=2,
		basicstyle=\tiny,
		captionpos=b,language=JAVA,breaklines=true,      % the size of the fonts that are used for the line-numbers,
		stepnumber=5,   
		keywordstyle=\color{brown},
		commentstyle=\color{DarkGreen}, 
		stringstyle=\color{blue},
		showstringspaces=false,
		breaklines=true
		literate=%
		{Ö}{{\"O}}1
		{Ä}{{\"A}}1
		{Ü}{{\"U}}1
		{ß}{{\ss}}1
		{ü}{{\"u}}1
		{ä}{{\"a}}1
		{ö}{{\"o}}1
		{~}{{\textasciitilde}}1	
	}

	\beamersetuncovermixins{\opaqueness<1>{25}}{\opaqueness<2->{15}}
	
	\usecaptiontemplate{
	\tiny
	\structure{\insertcaptionname~\insertcaptionnumber:}
	\insertcaption
	}
	
	\usefootnotetemplate{
	\tiny
	\parindent 1em\noindent
	\hbox to 1.8em{\hfil\insertfootnotemark}\insertfootnotetext
	}
	\begin{document}
			
	\setbeamercovered{invisible}
	
	\title[Review - State Machine Projekt]{Review - State Machine Projekt\\ Eine kollisionsfreie Robotersteuerung}
	\author{Gruppe 3}
	
	 \begin{frame}[title=Hauptgebaeude_Nacht.jpg]
	 	\maketitle
	 	\date{26. Mai 2018}
 	\end{frame}
	 
	\begin{frame}
		\frametitle{Gliederung}
		\tableofcontents
		%Folien mit einem * im Titel sollen beim Vortrag übersprungen werden.
	\end{frame}
	\section{RobotHandler (Server)}
	\begin{frame}{RobotHandler (Server)}
	\centering
		\includegraphics[height=0.75\textheight]{PDF/RobotHandler.pdf}
	\end{frame}
	\begin{frame}{RobotHandler (Server) - Detailansicht}
	\includegraphics[height=0.75\textheight]{PDF/RobotHandler1.pdf}
	\end{frame}	
\begin{frame}{RobotHandler (Server) - Detailansicht}
\includegraphics[height=0.75\textheight]{PDF/RobotHandler2.pdf}
\end{frame}	
\begin{frame}{RobotHandler (Server) - Detailansicht}
\includegraphics[height=0.75\textheight]{PDF/RobotHandler3.pdf}
\end{frame}	
\begin{frame}{RobotHandler (Server) - Detailansicht}
\includegraphics[width=\linewidth]{PDF/RobotHandler4.pdf}
\end{frame}	
	\section{TaskProcessing (Roboter)}
	\begin{frame}{TaskProcessing (Roboter)}
	\centering
	\includegraphics[height=0.75\textheight]{PDF/TaskProcessing.pdf}
	\end{frame}
\begin{frame}{TaskProcessing (Roboter) - Detailansicht}
\centering
\includegraphics[width=0.8\linewidth]{PDF/TaskProcessing1.pdf}
\end{frame}
\begin{frame}{TaskProcessing (Roboter) - Detailansicht}
\centering
\includegraphics[width=\linewidth]{PDF/TaskProcessing2.pdf}
\end{frame}

\section{DriveSystem (Roboter)}
\begin{frame}{DriveSystem (Roboter) - Übersicht (abstrakt)}
\centering
\includegraphics[height=0.75\textheight]{PDF/MiniDriveSystem.pdf}
\end{frame}
\begin{frame}{DriveSystem (Roboter) - Übersicht (detailliert)}
\centering
\includegraphics[height=0.75\textheight]{PDF/DriveSystem.pdf}
\end{frame}
\begin{frame}{DriveSystem (Roboter) - Detailansicht}
\centering
\includegraphics[height=0.75\textheight]{PDF/DriveSystem1.pdf}
\end{frame}
\begin{frame}{DriveSystem (Roboter) - Detailansicht}
\centering
\includegraphics[height=0.75\textheight]{PDF/DriveSystem2.pdf}
\end{frame}
\begin{frame}{DriveSystem (Roboter) - Detailansicht}
\centering
\includegraphics[height=0.75\textheight]{PDF/DriveSystem1.pdf}
\end{frame}
\begin{frame}{DriveSystem (Roboter) - Detailansicht}
\centering
\includegraphics[width=\linewidth]{PDF/DriveSystem4.pdf}
\end{frame}
\begin{frame}{DriveSystem (Roboter) - Detailansicht}
\centering
\includegraphics[height=0.75\textheight]{PDF/DriveSystem5.pdf}
\end{frame}
\begin{frame}{DriveSystem (Roboter) - Detailansicht}
\centering
\includegraphics[width=\linewidth]{PDF/DriveSystem6.pdf}
\end{frame}
\begin{frame}{DriveSystem (Roboter) - Detailansicht}
\centering
\includegraphics[height=0.75\textheight]{PDF/DriveSystem7.pdf}
\end{frame}
\begin{frame}{DriveSystem (Roboter) - Detailansicht}
\centering
\includegraphics[width=\linewidth]{PDF/DriveSystem8.pdf}
\end{frame}
\begin{frame}{DriveSystem (Roboter) - Detailansicht}
\centering
\includegraphics[width=\linewidth]{PDF/DriveSystem9.pdf}
\end{frame}
\section{Umgang mit Verklemmungen}
\begin{frame}{Umgang mit Verklemmungen}
\centering
\includegraphics[height=0.75\textheight]{PDF/Deadlocks1.pdf}
\end{frame}\begin{frame}{Umgang mit Verklemmungen}
\centering
\includegraphics[height=0.75\textheight]{PDF/Deadlocks2.pdf}
\end{frame}
	\begin{frame}[title=Hauptgebaeude_Nacht.jpg]
		\maketitle
		\date{26. Mai 2018}
	\end{frame}
\end{document}

