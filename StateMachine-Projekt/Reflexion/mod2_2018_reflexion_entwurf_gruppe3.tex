%% LyX 2.2.3 created this file.  For more info, see http://www.lyx.org/.
%% Do not edit unless you really know what you are doing.
\documentclass[11pt,ngerman,intoc]{article}
\usepackage[T1]{fontenc}
\usepackage[latin9]{inputenc}
\usepackage{geometry}
\geometry{verbose,tmargin=3cm,bmargin=2cm,lmargin=2cm,rmargin=2.3cm}
\setcounter{secnumdepth}{5}
\setcounter{tocdepth}{5}
\usepackage{float}
\usepackage{amsmath}
\usepackage{setspace}
\onehalfspacing

\makeatletter
%%%%%%%%%%%%%%%%%%%%%%%%%%%%%% Textclass specific LaTeX commands.
\numberwithin{equation}{section}
\numberwithin{table}{section}
\numberwithin{figure}{section}

%%%%%%%%%%%%%%%%%%%%%%%%%%%%%% User specified LaTeX commands.
\RequirePackage[ngerman=ngerman-x-latest]{hyphsubst}
\usepackage{caption}
\usepackage{lmodern}
\usepackage[bottom]{footmisc}
\usepackage{setspace}
\usepackage[natbibapa]{apacite} 
\onehalfspacing
\hyphenation{Dreh-mo-men-ten-gleich-ge-wicht} 
\newtheorem{defi}{Definition}[section]
\usepackage{url}
\usepackage{ucs}
\PrerenderUnicode{������}
\makeatletter
\newcommand{\tabitem}{~~\llap{\textbullet}~~}
\usepackage{pdfpages}
\usepackage{acronym}
\usepackage{graphicx}
\usepackage[colorlinks=false,
pdfpagelabels,
pdfstartview = FitH,
bookmarksopen = true,
bookmarksnumbered = true,
linkcolor = black,
plainpages = false,
hypertexnames = false,
citecolor = black] {hyperref}
\setlength{\parindent}{0pt}
\PassOptionsToPackage{hyphens}{url}
\usepackage{hyperref}
\usepackage{scrpage2}
\pagestyle{scrheadings}
\clearscrheadfoot
\chead[]{Modellierung II - SoSe 2018 - Reflexion des State Machine Projekts - Gruppe 3}
\cfoot[\pagemark]{\pagemark}
\expandafter\def\expandafter\UrlBreaks\expandafter{\UrlBreaks% save the current one
  \do\a\do\b\do\c\do\d\do\e\do\f\do\g\do\h\do\i\do\j%
  \do\k\do\l\do\m\do\n\do\o\do\p\do\q\do\r\do\s\do\t%
  \do\u\do\v\do\w\do\x\do\y\do\z\do\A\do\B\do\C\do\D%
  \do\E\do\F\do\G\do\H\do\I\do\J\do\K\do\L\do\M\do\N%
  \do\O\do\P\do\Q\do\R\do\S\do\T\do\U\do\V\do\W\do\X%
  \do\Y\do\Z\do\*\do\-\do\~\do\'\do\"\do\-}%

\makeatother

\usepackage{babel}
\begin{document}
	
\includepdf{Deckblatt.pdf}
\setcounter{page}{2}
\setcounter{section}{15}

\section{UML Zustandsdiagramme: Zust�nde}
Die grunds�tzliche Notation der Sequenzdiagramme bereitete uns wenig Probleme. Da wir die verschiedenen Sequenzdiagramme in der Gruppe aufteilten, war ein extra Schritt n�tig um iterierend immer wieder einheitliche Benennungen einzuf�hren. Hierf�r w�ren vermutlich zu Beginn festgelegte Richtlinien praktisch gewesen. Da die Sequenzdiagramme zu Beginn des Entwurfs stehen, lohnte es sich sehr an dieser Stelle schon die (gut versteckten) Visual Paradigm Features f�r Verkn�pfungen zwischen den Diagrammen einzustellen. Relativ viel, im Ergebnis unn�tiger, Aufwand entstand au�erdem dadurch, dass wir am Anfang Sequenzdiagramme auf einer zu niedrigen Abstraktionsebene modelliert haben.

\section{UML Zustandsdiagramme: Transitionen}
Mit der Verwendung von Klassendiagrammen f�r die genannten Anforderungen hatten wir kaum Schwierigkeiten. Das einzige wirkliche Problem f�r uns war, die Verbindung von Klassendiagrammen und den Komponentenschnittstellen zu verstehen. Nach der Erkl�rung, dass diese letztlich das selbe auf einer anderen Abstraktionsebene aussagen, war dieses Problem jedoch gel�st.

\section{UML Aktivit�tsdiagramme f�r detaillierte Abl�ufe}
Da unsere L�sung ohne zus�tzliche Aktivit�tsdiagramme ausgekommen ist, verzichten wir an dieser Stelle auf eine umfassende Reflexion. Das Feedback im Review hat jedoch gezeigt, dass es vermutlich an einigen Stellen sinnvoll gewesen w�re, kompliziertere Abl�ufe in einer Funktion zu kapseln die in einem Aktivit�tsdiagramm �ber Selbstaufrufe dargestellt wird.

\section{UML Klassendiagramme: Aktive Klassen und Enumerations}
In unserer Gruppe gab es anfangs Probleme bei dem finden einer passende Aufteilung der Klassen in einzelne Pakete, allerdings wurde im zweiten Versuch eine sinnvolle Aufteilung gefunden. \\
Auch bei der Benutzung von access und import gab es einige Ungewissheiten. Ein Teil davon wurde innerhalb der Gruppe gekl�rt und das Diagramm wurde dementsprechend korrigiert. Der andere Teil, welcher die Verwendung von access in Verbindung mit Interfaces war, wurde dann im Review gekl�rt.

\section{Modellierungswerkzeuge, Teamwork und Kommunikation}
Zun�chst ist zu sagen, dass wir die Aufgabenstellung und deren Komplexit�t zun�chst sehr untersch�tzt haben. Anschlie�end ist uns jedoch der Sinn der Reihenfolge der Aufgaben gel�ufiger geworden und wir konnten uns besser in die Objektorientierung einarbeiten. Zum Ende hin war daher die Arbeit deutlich effektiver, auch wenn dadurch die Arbeitszeit pro Woche angestiegen ist.

\end{document}
