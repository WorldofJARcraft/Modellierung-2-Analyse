%% LyX 2.2.3 created this file.  For more info, see http://www.lyx.org/.
%% Do not edit unless you really know what you are doing.
\documentclass[11pt,ngerman,intoc]{article}
\usepackage[T1]{fontenc}
\usepackage[latin9]{inputenc}
\usepackage{geometry}
\geometry{verbose,tmargin=3cm,bmargin=2cm,lmargin=2cm,rmargin=2.3cm}
\setcounter{secnumdepth}{5}
\setcounter{tocdepth}{5}
\usepackage{float}
\usepackage{amsmath}
\usepackage{setspace}
\onehalfspacing

\makeatletter
%%%%%%%%%%%%%%%%%%%%%%%%%%%%%% Textclass specific LaTeX commands.
\numberwithin{equation}{section}
\numberwithin{table}{section}
\numberwithin{figure}{section}

%%%%%%%%%%%%%%%%%%%%%%%%%%%%%% User specified LaTeX commands.
\RequirePackage[ngerman=ngerman-x-latest]{hyphsubst}
\usepackage{caption}
\usepackage{lmodern}
\usepackage[bottom]{footmisc}
\usepackage{setspace}
\usepackage[natbibapa]{apacite} 
\onehalfspacing
\hyphenation{Dreh-mo-men-ten-gleich-ge-wicht} 
\newtheorem{defi}{Definition}[section]
\usepackage{url}
\usepackage{ucs}
\PrerenderUnicode{������}
\makeatletter
\newcommand{\tabitem}{~~\llap{\textbullet}~~}
\usepackage{pdfpages}
\usepackage{acronym}
\usepackage{graphicx}
\usepackage[colorlinks=false,
pdfpagelabels,
pdfstartview = FitH,
bookmarksopen = true,
bookmarksnumbered = true,
linkcolor = black,
plainpages = false,
hypertexnames = false,
citecolor = black] {hyperref}
\setlength{\parindent}{0pt}
\PassOptionsToPackage{hyphens}{url}
\usepackage{hyperref}
\usepackage{scrpage2}
\pagestyle{scrheadings}
\clearscrheadfoot
\chead[]{Modellierung II - SoSe 2018 - Reflexion des State Machine Projekts - Gruppe 3}
\cfoot[\pagemark]{\pagemark}
\expandafter\def\expandafter\UrlBreaks\expandafter{\UrlBreaks% save the current one
  \do\a\do\b\do\c\do\d\do\e\do\f\do\g\do\h\do\i\do\j%
  \do\k\do\l\do\m\do\n\do\o\do\p\do\q\do\r\do\s\do\t%
  \do\u\do\v\do\w\do\x\do\y\do\z\do\A\do\B\do\C\do\D%
  \do\E\do\F\do\G\do\H\do\I\do\J\do\K\do\L\do\M\do\N%
  \do\O\do\P\do\Q\do\R\do\S\do\T\do\U\do\V\do\W\do\X%
  \do\Y\do\Z\do\*\do\-\do\~\do\'\do\"\do\-}%

\makeatother

\usepackage{babel}
\begin{document}
	
\includepdf{Deckblatt.pdf}
\setcounter{page}{2}
\setcounter{section}{15}

\section{UML Zustandsdiagramme: Zust�nde}
Da uns das Prinzip des Automaten mit Zust�nden bereits aus Modellierung 1 bekannt war, hatten wir an dieser Stelle keine fundamentalen Schwierigkeiten.\\
Ein bedeutender Fehler war, den Startzustand als einen Zustand anzusehen, in den man zur�ckkehren kann. Au�erdem h�tten wir vielleicht etwas mehr mit entrys und der Do-Methode des Zustandes arbeiten sollen, etwa um das Warten auf eine Antwort darzustellen.

\section{UML Zustandsdiagramme: Transitionen}
Die grundlegenden M�glichkeiten die zum Ausl�sen von Transitionen in der UML zur Verf�gung stehen, bereiteten wenig Probleme.
Viele m�chtigere M�glichkeiten waren uns beim L�sen der Aufgabe jedoch nicht bekannt (insbesondere when-Statements). Hier h�tten wir uns nach Verantwortlichkeiten noch tiefer in die UML einarbeiten sollen.\\
Ein �hnliches Problem schafften wir uns mit Verkettungen von Entscheidungen, in denen Seiteneffekte auf Variablen auftraten, die zu einem sp�teren Zeitpunkt als Bedingung fungierten. Hier h�tten mehr Zust�nde geholfen.\\
Wir h�tten uns au�erdem eine detailliertere Auseinandersetzung mit solchen komplexeren M�glichkeiten im Rahmen der Vorlesung gew�nscht.


\section{UML Aktivit�tsdiagramme f�r detaillierte Abl�ufe}
Da unsere L�sung ohne zus�tzliche Aktivit�tsdiagramme ausgekommen ist, verzichten wir an dieser Stelle auf eine umfassende Reflexion. Das Feedback im Review hat jedoch gezeigt, dass es vermutlich an einigen Stellen sinnvoll gewesen w�re, kompliziertere Abl�ufe in einer Funktion zu kapseln die in einem Aktivit�tsdiagramm �ber Selbstaufrufe dargestellt wird.

\section{UML Klassendiagramme: Aktive Klassen und Enumerations}
Mit der Verwendung von Enumerations hatten wir in keiner Weise Probleme, da diese bereits aus der Einf�hrungsveranstaltung in die Programmiertechnik bekannt waren. Wir empfanden sie als sinnvolle Erg�nzung zu den Methodenaufrufen und dem durch den aktuellen Zustand dargestellten Zustand des Automaten.\\
Mit aktiven Klassen haben wir nicht gearbeitet, da unserer Meinung nach mit dem Zustandsautomaten das erw�nschte Verhalten eindeutig dargestellt werden konnte.

\section{Modellierungswerkzeuge, Teamwork und Kommunikation}
Da unser altbew�rtes Tool Visual Paradigm keine besonders gute Implementation von State Machines bereitstellte haben wir uns f�r die Umsetzung mittels des Online-Tools draw.io entschieden. Da zu den Terminen schon sehr ausgereifte Modelle vorhanden waren, beschr�nkte sich die Kommunikation auf die gemeinsamen Treffen, in denen wir die Modelle und Vortr�ge finalisieren konnten.


\end{document}
